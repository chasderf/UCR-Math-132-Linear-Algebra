\documentclass{article}

\begin{document}

\textbf {Chapter 8: Eigenvalues: Further Applications and Computations}

\textbf {Chapter 8.1 Diagonalization of Quadratic Forms}

\textit {Every quadratic form in n variables $x_i$ can be written as $x^T U$x where x is the column vector of variables and U is a nonzero upper-triangular matrix}

\textit {Every quadratic form in n variables $x_i$ can be written as $x^T A$x, where x is the column vector of variables and A is a symmetric matrix.}

\textbf {Theorem 8.1 Principal Axis Theorem} Every quadratic form f(x) in n variables $x_1, x_2, ... , x_n$ can be diagonalized by a substituion x = Ct, where C is an n X n orthogonal matirx. The diagonalized form appears as

\begin{center}
$\lambda_1 {\lambda_1}^2 + \lambda_2 {\lambda_2}^2 + ... + \lambda_n {\lambda_n}^2$
\end{center}

where the $\lambda_j$ are the eigenvalues of the symmetric coefficient matrix A of f(x). The jth column vector of C is a normalized eigenvector $v_j$ of A corresponding to $\lambda_j$. Moreover, C can be chosen so that det(C) = 1










\end{document}