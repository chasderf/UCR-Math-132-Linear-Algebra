\documentclass {article}

\begin{document}

\textbf {Chapter 9 Complex Sccalars}

\textit {Fundamental Theorem of Algebra} Every polynomial equation with coefficients in C has n solutions in C, where n is the degree of the polynomial and the solutions are counted with their algebraic multiplicity.
\\

The modulus (or magnitude) of the complex number z = a + bi is $\mid z \mid = \sqrt{a^2 + b^2}$ which is the length of the vector in Figure 9.1.

\textit { Geometric Representation of $z_1 z_2$ }
\\
1. $ \mid z_1 z_2 \mid = \mid z_1 \mid \mid z_2 \mid$
\\
2. Arg($z_1$) + Arg($z_2$) is an argument of $z_1 z_2$.

\textit { Geometric Representation of $z_1 / z_2$ }
\\
1. $ \mid z_1 / z_2 \mid = \mid z_1 \mid / \mid z_2 \mid$
\\
2. Arg($z_1$) + Arg($z_2$) is an argument of $z_1 z_2$.

nth roots of z = r(cos $\theta$ + i sin $\theta$) \\
The nth roots of z are

\begin{center}
$ r^{1/n} (cos( {\frac{\theta} {n}} + {\frac {2k\pi} {n}} ) + i sin( \frac{\theta} {n} + \frac {2k\pi}{n}))$
\end{center}

for k = 0, 1,2, ... , n - 1.

\textbf { Chapter 9.2 Matrices and Vector Spaces with Complex scalars }
\textbf {Definition 9.1 Euclidean inner product} Let u = {$u_1 , {u_2} ,..., {u_n}$} and v = [${v_1}, {v_2}, ... , {v_n}$] be vectors in $C^n$. The Euclidean inner product of u and v is 
\begin{center}
$ u, v = \bar{u_1} v_1 + \bar{u_2} v_2 + ... + \bar {u_n} v_n.$
\end{center}

\textbf {Theorem 9.2 Properties of the Euclidean Inner Product} \\
Let u, v and w be vectors in $c^n$, and let z be a complex scalar. \\

1. $\langle u, u \rangle \leq$ 0, and $\langle u , u \rangle$ = 0 if and only if u = 0 \\

2. $\langle u, v \rangle  = {\langle \bar{v, u} \rangle}$,
\\
3. $\langle (u + v), w \rangle = {\langle u, w \rangle} + \langle v, w \rangle , $
\\
4. $\langle w, (u + v) \rangle = {\langle w, u \rangle} + {\langle w, v \rangle},$
\\
5. $\langle zu , v \rangle = \bar{z} \langle u, v \rangle, and \langle u, zv \rangle = z \langle u, v \rangle $

\textit {The Euclidean inner product in $C^n$ is not commutative}

\textbf {Definition 9.2 Conjugate Transpose or Hermitian Adjoint} Let A = [$a_{ij}$] be an m X n matrix with complex scalar entries. \\

1. The conjugate of A is the m X n matrix $\bar{A} = [\bar{a{ij}}]$. \\

2. The conjugate transpose (or Hermitian adjoint) of A is the matrix $A^* = [\bar{a_ij}]^T$. 

\textbf {Theorem 9.3 Properties of the Conjugate Transpose} \\
Let A and B be m X n matrices. Then \\
1. $(A^*)^* = A$ \\
2. $(A + B)^* = A^* + B^*$, \\
3. $(zA)^* = \bar{z}A^*$ for any scalar $ z \in C, $ \\
4. If A and B are square matrices, $(AB)^* = B^*A^*$

\textbf {Definition 9.3 Unitary Matrix} \\
A square matrix U with complex entries in unitary if its column vectors are orthogonal unit vectors - that is, if $U^*U=I$

\textbf {Eigenvalues and diagonalization}

\textit {Every real symmetric matrix is diagonalizable by a real orthogonal matrix.}

\textit {Every Hermitian matrix is diagonalizable by a unitary matrix }

\textbf {Theorem 9.4 Schur's Lemma} \\
Let A be an n X n (complex) matrix. There is a unitary matrix U such that $U^{-1}AU$ is upper trangular.

\textbf {Theorem 9.5 Spectral Theorem for Hermitian Matrices} \\
If A is a Hermitian matrix, there exists a unitary matrix U such that $U^{-1}AU$ is a diagonal matrix. Furthermore, all eigenvalues of A are real.

\textit {Corollary Fundamental Theorem of Real Symmetric Matrices} \\
Every n X n real symmetric matrix has n real eigenvalues, counted with their algebraic multiplicity, and is diagonalizable by a real orthogonal matrix.

\textbf {Theorem 9.6 Orthogonality of Eigenspaces of a Hermitian Matrix} \\
The eigenvectors of a Hermitian matrix corresponding to distinct eigenvalues are orthogonal.

\textbf {Definition 9.5 Normal Matrix} \\
A square matrix A is normal if it commutes with its conjugate transpose, so that $AA^* = A^*A$.

\textbf {Theorem 9.7 Spectral Theorem for Normal Matrices} \\ 
A square matrix A is unitarily diagonalizable if and only if it is a normal matrix.

\textbf {Chapter 9.4 Jordan Canonical Form}

\textbf {Definition 9.6 Jordan Block} \\
An m X m matrix is a Jordan block if it is structured as follows: \\
1. All diagonal entries are equal \\
2. Each entry immediately above a diagonal entry is 1. \\
3. All other entries are zero.

\textbf {Theorem 9.8 Properties of a Jordan Block} \\
Let J be an m X m Jordan block with diagonal entries all equal to $\lambda$. Then the following properties hold \\
1. (J - $\lambda I)e_i$ = $e_{i-1}$ for $1 < i \leq m$, and $(J - \lambda I)e_1 = 0.$ \\

2. (J - $\lambda I)^m = O, but (J - \lambda I)^i \neq O for i < m.$ \\
3. $Je_i = \lambda e_i + e_{i-1} for 1 < i \leq m,$ whereas $Je_1 = \lambda e_1$

\textbf {Definition 9.7 Jordan Canonical Form} An n X n matrix J is a Jordan canonical form if it consists of Jordan blocks, placed corner to corner along the main diagonal, as in matrix (4) with only zero entries outside these Jordan blocks.

\textbf {Definition 9.8 Jordan Basis} \\
Let A be an n X n matrix. An ordered basis B = $(b_1,b_2,...,b_n)$ of $C^n$ is a Jordan basis for A if, for 1 $\leq j \leq n$, we have either $Ab_j = \lambda b_j$ or $Ab_j = \lambda b_j + b_{j-1}$, where $\lambda$ is an eigenvalue of A that we say is associated with $b_j$. If $Ab_j = \lambda b_j + b_{j-1}$, we require that the eigenvalue associated with $b_{j-1}$ also be $\lambda$.

\textbf {Theorem 9.9 Jordan Canonical Form of a Square Matrix} \\
Let A be a square matrix. There exists an invertible matrix C such that the matrix J = $C^{-1}AC$ is a Jordan Canonical form. This Jordan canonical form is unique, except for the order of the Jordan bloacks of which it is composed.









\end{document}